\documentclass[a4paper,11pt]{article}
\usepackage{fullpage}
\usepackage[latin1]{inputenc}
\usepackage[T1]{fontenc}
\usepackage[normalem]{ulem}
\usepackage[english]{babel}
\usepackage{listings,babel}
\lstset{breaklines=true,basicstyle=\ttfamily}
\usepackage{graphicx}
\usepackage{moreverb}
\usepackage{url}

\title{Wishbone Block RAM}
\author{S\'ebastien Bourdeauducq}
\date{December 2009}
\begin{document}
\setlength{\parindent}{0pt}
\setlength{\parskip}{5pt}
\maketitle{}
\section{Specifications}
This core creates 32-bit storage RAM on the Wishbone bus by using FPGA Block RAM.

Byte-wide writes are supported. Burst access is not supported.

The typical use case is to provide initial memory for softcore CPUs.

\section{Using the core}
You should specify the block RAM storage depth, in bytes, by using the \verb!adr_width! parameter.

\section*{Copyright notice}
Copyright \copyright 2007-2009 S\'ebastien Bourdeauducq. \\
Permission is granted to copy, distribute and/or modify this document under the terms of the GNU Free Documentation License, Version 1.3; with no Invariant Sections, no Front-Cover Texts, and no Back-Cover Texts. A copy of the license is included in the LICENSE.FDL file at the root of the Milkymist source distribution.

\end{document}
